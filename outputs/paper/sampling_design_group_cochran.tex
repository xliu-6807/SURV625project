% Options for packages loaded elsewhere
\PassOptionsToPackage{unicode}{hyperref}
\PassOptionsToPackage{hyphens}{url}
\PassOptionsToPackage{dvipsnames,svgnames,x11names}{xcolor}
%
\documentclass[
  12pt]{article}

\usepackage{amsmath,amssymb}
\usepackage{iftex}
\ifPDFTeX
  \usepackage[T1]{fontenc}
  \usepackage[utf8]{inputenc}
  \usepackage{textcomp} % provide euro and other symbols
\else % if luatex or xetex
  \usepackage{unicode-math}
  \defaultfontfeatures{Scale=MatchLowercase}
  \defaultfontfeatures[\rmfamily]{Ligatures=TeX,Scale=1}
\fi
\usepackage{lmodern}
\ifPDFTeX\else  
    % xetex/luatex font selection
\fi
% Use upquote if available, for straight quotes in verbatim environments
\IfFileExists{upquote.sty}{\usepackage{upquote}}{}
\IfFileExists{microtype.sty}{% use microtype if available
  \usepackage[]{microtype}
  \UseMicrotypeSet[protrusion]{basicmath} % disable protrusion for tt fonts
}{}
\makeatletter
\@ifundefined{KOMAClassName}{% if non-KOMA class
  \IfFileExists{parskip.sty}{%
    \usepackage{parskip}
  }{% else
    \setlength{\parindent}{0pt}
    \setlength{\parskip}{6pt plus 2pt minus 1pt}}
}{% if KOMA class
  \KOMAoptions{parskip=half}}
\makeatother
\usepackage{xcolor}
\setlength{\emergencystretch}{3em} % prevent overfull lines
\setcounter{secnumdepth}{5}
% Make \paragraph and \subparagraph free-standing
\makeatletter
\ifx\paragraph\undefined\else
  \let\oldparagraph\paragraph
  \renewcommand{\paragraph}{
    \@ifstar
      \xxxParagraphStar
      \xxxParagraphNoStar
  }
  \newcommand{\xxxParagraphStar}[1]{\oldparagraph*{#1}\mbox{}}
  \newcommand{\xxxParagraphNoStar}[1]{\oldparagraph{#1}\mbox{}}
\fi
\ifx\subparagraph\undefined\else
  \let\oldsubparagraph\subparagraph
  \renewcommand{\subparagraph}{
    \@ifstar
      \xxxSubParagraphStar
      \xxxSubParagraphNoStar
  }
  \newcommand{\xxxSubParagraphStar}[1]{\oldsubparagraph*{#1}\mbox{}}
  \newcommand{\xxxSubParagraphNoStar}[1]{\oldsubparagraph{#1}\mbox{}}
\fi
\makeatother


\providecommand{\tightlist}{%
  \setlength{\itemsep}{0pt}\setlength{\parskip}{0pt}}\usepackage{longtable,booktabs,array}
\usepackage{calc} % for calculating minipage widths
% Correct order of tables after \paragraph or \subparagraph
\usepackage{etoolbox}
\makeatletter
\patchcmd\longtable{\par}{\if@noskipsec\mbox{}\fi\par}{}{}
\makeatother
% Allow footnotes in longtable head/foot
\IfFileExists{footnotehyper.sty}{\usepackage{footnotehyper}}{\usepackage{footnote}}
\makesavenoteenv{longtable}
\usepackage{graphicx}
\makeatletter
\def\maxwidth{\ifdim\Gin@nat@width>\linewidth\linewidth\else\Gin@nat@width\fi}
\def\maxheight{\ifdim\Gin@nat@height>\textheight\textheight\else\Gin@nat@height\fi}
\makeatother
% Scale images if necessary, so that they will not overflow the page
% margins by default, and it is still possible to overwrite the defaults
% using explicit options in \includegraphics[width, height, ...]{}
\setkeys{Gin}{width=\maxwidth,height=\maxheight,keepaspectratio}
% Set default figure placement to htbp
\makeatletter
\def\fps@figure{htbp}
\makeatother

\addtolength{\oddsidemargin}{-.5in}%
\addtolength{\evensidemargin}{-1in}%
\addtolength{\textwidth}{1in}%
\addtolength{\textheight}{1.7in}%
\addtolength{\topmargin}{-1in}%
\usepackage{booktabs}
\usepackage{longtable}
\usepackage{array}
\usepackage{multirow}
\usepackage{wrapfig}
\usepackage{float}
\usepackage{colortbl}
\usepackage{pdflscape}
\usepackage{tabu}
\usepackage{threeparttable}
\usepackage{threeparttablex}
\usepackage[normalem]{ulem}
\usepackage{makecell}
\usepackage{xcolor}
\makeatletter
\@ifpackageloaded{caption}{}{\usepackage{caption}}
\AtBeginDocument{%
\ifdefined\contentsname
  \renewcommand*\contentsname{Table of contents}
\else
  \newcommand\contentsname{Table of contents}
\fi
\ifdefined\listfigurename
  \renewcommand*\listfigurename{List of Figures}
\else
  \newcommand\listfigurename{List of Figures}
\fi
\ifdefined\listtablename
  \renewcommand*\listtablename{List of Tables}
\else
  \newcommand\listtablename{List of Tables}
\fi
\ifdefined\figurename
  \renewcommand*\figurename{Figure}
\else
  \newcommand\figurename{Figure}
\fi
\ifdefined\tablename
  \renewcommand*\tablename{Table}
\else
  \newcommand\tablename{Table}
\fi
}
\@ifpackageloaded{float}{}{\usepackage{float}}
\floatstyle{ruled}
\@ifundefined{c@chapter}{\newfloat{codelisting}{h}{lop}}{\newfloat{codelisting}{h}{lop}[chapter]}
\floatname{codelisting}{Listing}
\newcommand*\listoflistings{\listof{codelisting}{List of Listings}}
\makeatother
\makeatletter
\makeatother
\makeatletter
\@ifpackageloaded{caption}{}{\usepackage{caption}}
\@ifpackageloaded{subcaption}{}{\usepackage{subcaption}}
\makeatother

\ifLuaTeX
  \usepackage{selnolig}  % disable illegal ligatures
\fi
\usepackage[]{natbib}
\bibliographystyle{agsm}
\usepackage{bookmark}

\IfFileExists{xurl.sty}{\usepackage{xurl}}{} % add URL line breaks if available
\urlstyle{same} % disable monospaced font for URLs
\hypersetup{
  pdftitle={Michigan Teen Smoking and Drug Use Survey Sample Design},
  pdfauthor={Kevin Linares; Jianing Zou; Weishan Jiang; Xiaoqing Liu},
  colorlinks=true,
  linkcolor={blue},
  filecolor={Maroon},
  citecolor={Blue},
  urlcolor={Blue},
  pdfcreator={LaTeX via pandoc}}



\begin{document}


\def\spacingset#1{\renewcommand{\baselinestretch}%
{#1}\small\normalsize} \spacingset{1}


%%%%%%%%%%%%%%%%%%%%%%%%%%%%%%%%%%%%%%%%%%%%%%%%%%%%%%%%%%%%%%%%%%%%%%%%%%%%%%

\date{April 23, 2025}
\title{\bf Michigan Teen Smoking and Drug Use Survey Sample Design}
\author{
Kevin Linares\\
and\\Jianing Zou\\
and\\Weishan Jiang\\
and\\Xiaoqing Liu\\
University of Maryland\\
}
\maketitle

\bigskip
\bigskip
\begin{abstract}
The State of Michigan Department of Education (MDE) requires data to
monitor teenage smoking and drug use to assess compliance with tobacco
industry settlements. This report outlines a two-stage stratified
cluster sampling design developed for the MDE to monitor smoking and
drug use among students in grades 7-12 statewide. The design addresses
the need for cost-effective, statistically sound estimates at both the
state and regional levels, meeting specified precision targets (CV=0.05)
within a \$500,000 budget. Schools were stratified into nine regions and
selected with probability proportional to size (PPeS), followed by
systematic selection of students within schools. The final design
targets approximately 88 schools and 4,721 students, incorporating
adjustments for non-response and procedures for handling undersized
schools to maintain equal selection probabilities. The report details
allocation, selection methods, and an estimation plan using a paired
selection model for variance calculation.
\end{abstract}


\newpage
\spacingset{1.9} % DON'T change the spacing!


\section{Introduction}\label{sec-intro}

The State of Michigan Department of Education (MDE) is specifically
interested in three outcome variables; ever smoked one cigarette, every
smoked marijuana, and age when first approach to smoke cigarettes or
marijuana. Moreover, MDE provided us with expected levels of precision
means and coefficient of variation (CV=0.05) shown in Table 1.

\begin{longtable}[]{@{}cccc@{}}
\caption{Key Variables and Desired Levels of Precision}\tabularnewline
\toprule\noalign{}
Outcome & Type & Desired CV & Expected Mean \\
\midrule\noalign{}
\endfirsthead
\toprule\noalign{}
Outcome & Type & Desired CV & Expected Mean \\
\midrule\noalign{}
\endhead
\bottomrule\noalign{}
\endlastfoot
smoked\_cig & prop & 0.05 & 0.25 \\
smoked\_mj & prop & 0.05 & 0.15 \\
age\_approached & mean & 0.05 & 12.00 \\
\end{longtable}

From the desired levels of precision, we can compute the desired simple
random sample (SRS) sample sizes when CV =.05 as
\(N = \frac{s^2}{se^2}\). We first must calculate the element variance
for each key variable. For proportions we use \(\hat{p}(1-\hat{p})\),
while for age we square the estimated standard deviation of 1,
\(v(\bar{y}) = \sigma^2\). We then calculate the standard error as
\(se(\hat{p}) = CV \times \hat{p}\). Finally, we estimate the desired
sampling variance as
\(var(\hat{p}) =  se(\hat{p})^2 \text{, where }  se(\hat{p}) = \sqrt{var(\hat{p})}\).
We show these results in Appendix 1, and note that these desired levels
of precision would lead to large differences in sample sizes for each
target variable. Therefore, we may wish to consider a more complex
survey design.

\section{Sampling Design}\label{sec-meth}

We employ a two-stage stratified cluster sampling design as it improve
sample efficiency and representation by ensuring subgroups are included
(i.e., stratification) while reducing costs and logistical challenges
associated with sampling individuals across large geographic areas. In
the first stage, random selection of schools (e.g., Primary Sampling
Units {[}PSU{]}) are determined by proportionate allocation for each
strata. The second stage randomly selects students (Secondary Selection
Units {[}SSU{]}) from the selected school clusters within each region.

The MDE provided the 2024 7th through 12th grade student headcount for
each public and private school within each of the nine regions resulting
in a target frame of \(830,138\) across \(2,443\) schools (78\% Public).
Schools in the first stage will be selected with probability
proportional to student body size (PPeS). The proportionate allocation
\(M_h / \sum M_h\) where \(M_h\) is the total number of students in
stratum \(h\), will be used to determine school number selection in the
first stage. Appendix 2 presents for each of the 9 strata total
students, number of schools, and proportionate allocation.

We obtained design effects (DEFF) estimates (DEFF\_cig=2.5,
DEFF\_mj=2.0, DEFF\_age=1.7) from a similar pilot study of 7,500
students based on 150 schools with 50 students each, and based on these
we estimate the rate of homogeneity \(roh\) for each target variable. We
use the provided DEFFs to estimate \(roh\) as
\(\hat{roh} = \frac{DEFF-1}{m-1}\), where \(m\) is the total sampled
students in the pilot study, to consider alternative cluster sample
designs along with cost considerations. Appendix 3 provides the \(roh\)
estimate for each target variable.

\subsection{\texorpdfstring{\emph{Sampling Within
Budget}}{Sampling Within Budget}}\label{sampling-within-budget}

Recall that our budget cost constraints as the cost per cluster as
\(c_n = \$3,000\) and cost per student as \(c_m = \$50\), with a total
budget constraint of \(C = \$500,000\). We can use the \(roh\)estimates
along with these costs to estimate the optimum subsample size
\(m_{opt}\) needed to achieve the desired precision as
\(m_{opt} = \sqrt{\frac{c_n}{c_m} \frac{1-roh}{roh}}\). Note that since
we have three target variables we also have three separate \(roh\)
estimates and thus three \(m_{opt}\) estimates. Similarly, we can use
\(m_{opt}\) to estimate the number of schools \(n_{opt}\) to sample,
\(n_{opt} = \frac{C}{c_n + m_{opt} \times c_m}\) . Finally, we compute
new DEFF for each variable as \(roh\) is portable and since we already
computed \(m_{opt}\) as \(DEFF_{new} = 1 + (m_{opt} - 1) \times roh\).
By multiplying \(m_{opt} \times n_{opt}\) for each variable we also get
the total subsample size, as well as compute the total cost using
\(n_{opt} \times c_n + n_{opt} \times m_{opt} \times c_m\). Table 2
shows for each target variable \(m_{opt}\), \(n_{opt}\), \(DEFF_{new}\),
total subsample size denoted by \(total_{nm}\), and the total cost.
Notice that our \(DEFF_{new}\) estimates are close to those from the
pilot study since again we used these to compute \(roh\) which is
portable for estimating new design effects.

\begin{longtable}[]{@{}
  >{\centering\arraybackslash}p{(\columnwidth - 12\tabcolsep) * \real{0.2162}}
  >{\centering\arraybackslash}p{(\columnwidth - 12\tabcolsep) * \real{0.1351}}
  >{\centering\arraybackslash}p{(\columnwidth - 12\tabcolsep) * \real{0.1351}}
  >{\centering\arraybackslash}p{(\columnwidth - 12\tabcolsep) * \real{0.1351}}
  >{\centering\arraybackslash}p{(\columnwidth - 12\tabcolsep) * \real{0.1351}}
  >{\centering\arraybackslash}p{(\columnwidth - 12\tabcolsep) * \real{0.1351}}
  >{\centering\arraybackslash}p{(\columnwidth - 12\tabcolsep) * \real{0.1081}}@{}}
\caption{Estimating New DEFF and Total Cost}\tabularnewline
\toprule\noalign{}
\begin{minipage}[b]{\linewidth}\centering
Outcome
\end{minipage} & \begin{minipage}[b]{\linewidth}\centering
m\_opt
\end{minipage} & \begin{minipage}[b]{\linewidth}\centering
n\_opt
\end{minipage} & \begin{minipage}[b]{\linewidth}\centering
deff\_new
\end{minipage} & \begin{minipage}[b]{\linewidth}\centering
total\_nm
\end{minipage} & \begin{minipage}[b]{\linewidth}\centering
cost
\end{minipage} & \begin{minipage}[b]{\linewidth}\centering
Option
\end{minipage} \\
\midrule\noalign{}
\endfirsthead
\toprule\noalign{}
\begin{minipage}[b]{\linewidth}\centering
Outcome
\end{minipage} & \begin{minipage}[b]{\linewidth}\centering
m\_opt
\end{minipage} & \begin{minipage}[b]{\linewidth}\centering
n\_opt
\end{minipage} & \begin{minipage}[b]{\linewidth}\centering
deff\_new
\end{minipage} & \begin{minipage}[b]{\linewidth}\centering
total\_nm
\end{minipage} & \begin{minipage}[b]{\linewidth}\centering
cost
\end{minipage} & \begin{minipage}[b]{\linewidth}\centering
Option
\end{minipage} \\
\midrule\noalign{}
\endhead
\bottomrule\noalign{}
\endlastfoot
smoked\_cig & 43.58899 & 96.53536 & 2.303745 & 4207.879 & \$500,000 &
1 \\
smoked\_mj & 53.66563 & 87.97734 & 2.074809 & 4721.360 & \$500,000 &
2 \\
age\_approached & 64.34283 & 80.42281 & 1.904898 & 5174.631 & \$500,000
& 3 \\
\end{longtable}

\subsection{\texorpdfstring{\emph{Evaluating Alternative Clustering
Designs}}{Evaluating Alternative Clustering Designs}}\label{evaluating-alternative-clustering-designs}

We have the possability of three clustering design options to choose
from based on the target variables. Using the \(m_{opt}\) values from
Table 2 as our three options, we iterate over each set of target
variables using these values to recompute for each target variable a new
design effect and evaluate the estimated sampling variance for each
design. We estimate the SRS sampling variance as
\(var_{srs} = \frac{var}{total_{nm} - 1}\) where var is the sampling
variance calculated from the pilot study and the denominator is the
degrees of freedom. Additionally, we estimate the sampling variance for
the clustering design as \(var_{crs} = var_{srs} \times deff_{new}\).
After estimating \(var_{crs}\), we can square it to estimate a standard
error, \(se=\sqrt{var_{crs}}\) to use to estimate 95\% confidence
intervals for the estimated means. Additionally, in our evaluation of
\(m_{opt}\) we can determine if the estimated sampling variance from the
complex design is smaller than what is desired from MDE; therefore,
Appendix 4 shows sampling variances, standard errors, confidence
intervals, and a variance check (e.g., ``Y'' = yes if \textless= to
desired sampling variance) for each target variable using the
\(m_{opt}\) options.

We determine that option 2 has reasonable estimated design effects
comparable to those from the pilot study as well as it is the only
option to pass the estimated sampling variance check we designed. Option
2 with \(m_{opt} = 53, n_{opt} = 88\) would cost a total of \$497,200.
Since the total student population is 830138 and our now target sample
is 4721 we can estimate the sampling fraction to be \(f = n/N\) =
0.0056874. The sampling fraction is the ratio of the sample size to the
population size, and this estimate translates to the sample comprises
approximately 0.57\% of the total student population. In this case, the
sampling fraction is low and the finite population correction factor is
not needed for calculating variances to adjust for the fact that
sampling without replacement from a finite population reduces
variability compared to sampling from an infinite population.

\subsection{\texorpdfstring{\emph{Non-Response
Adjustments}}{Non-Response Adjustments}}\label{non-response-adjustments}

The MDE anticipates 30\% school response rates and 70\% among students;
therefore, we adjust the number of schools and within-school target by
dividing\(m_{opt} \times RR_{student} = 53.6766 \div .70\) = 76.6651878
students and for schools
\(n_{opt} \times RR_{schools} =  87.9689 \div .30\) = 293.2578028. We
use these values to allocate the number of clusters for each strata
based on the proportion allocated we calculated.

\section{Stage 1 Selection}\label{stage-1-selection}

We consider stratified PPeS selection of schools from each strata by
first sorting the list of schools to achieve implicit stratification. We
sort by taking the number of 9th through 12th grade for each school
divided by the total student body and descending order, and in this way
we hypothesize that schools with older students are more likely to be
positively associated with the target variables. For each strata \(h\)
we assign our adjusted \(n_{opt} \times proportionate \_ allocation\)
estimated earlier to calculate the number of schools to sample, denoted
as \(n_h\) in Appendix 5. We use \(n_h\) to calculated in this Table the
sampling interval \(k_h = \frac{\sum_{i \ in h} MOS_{hi}}{n_h}\) where
\(MOS_{hi}\) is the measure of size (MOS), total student head-count, for
each school \(i\) in strata \(h\). The \(k_h\) parameter is an important
component of systematic sampling to determine how frequently units are
selected from an ordered list. We randomly select a number between 1 and
\(k_h\) for selecting schools from the list, denoted as \(RN\).

For each strata we use the random start to select the first school, and
for stratum with more than one selection we use
\(RN, RN+k_h, RN+2k_h,...,RN +(n_h-1)\) until we satisfy \(n_h\)
selection. Our minimum MOS 76.67 is also our \(m_{opt}\) and we use it
here to determine the minimum number of students in each selected school
required and if this is not satisfied we perform post-selection linkage.
The linking is done by first selecting the number of schools in each
strata. When the next units on the list do not meet the sufficient MOS
required we move forward in the list until the first unit that meets the
minimum requirement is achieved. For all the units that did not meet the
requirement they are cumulated backwards until a linked unit of minimum
sufficient size is created. We do this process for all strata. Appendix
6 shows for each strata the total number of clusters, how many totaled
schools linked and the total number of students.

\section{Stage 2 Selection}\label{stage-2-selection}

We assume rosters are made available by the school administration at the
time of data collection. These rosters are ordered and formatted
uniformly to facilitate systematic sampling. To maintain equal
probability of selection (epsem) across all strata, we computed the
required number of students to be sampled per selected school, denoted
as \(m^*_h\) , based on the within-strata sampling fraction \(f_h\),
which is the same as the overall sampling fraction \(f\) for all \(h\),
and the stratum-specific PPS sampling interval \(k_h\) as follows,
\(f_h = \frac{n_h MOS_{hi} }{\sum_{i \in h} MOS{hi} } \frac{m^*_h}{MOS_{hi}} = \frac{n_h m^*_h}{\sum_{e \in h}MOS_{hi}} \Rightarrow m^*_h = f \times k_h\).
This ensures that when each school is selected with probability
proportional to its MOS and then students are sampled within school at a
fixed rate \(\frac{m^*_h}{MOS_{hi}}\), the overall inclusion probability
for any student is,
\(\pi_{i} = \frac{n_h \times MOS_{hi}}{\sum MOS_{hi}} \times \frac{m^*_h}{MOS_{hi}} = \frac{n_h m^*_h}{\sum MOS_h}\).
Each student has the same selection probability within the state and
within region \(h\), satisfying the epsem condition. Table 3 summarizes
the first stage stratification and selection for two strata. The
tolerated minimum number of students per school is estimated as
\(m^*_h\) divided by the expected student response rate of 0.70 and is
denoted as b\_h in the table. Additionally, we compute the sampling
interval for each stratum to achieve epsem, as well as show the random
start used to select schools.

\begin{longtable}[]{@{}ccccccc@{}}
\caption{Sampling Interval for Two Strata,}\tabularnewline
\toprule\noalign{}
Region & f\_h & MOS\_h & n\_h & b\_h & k\_h & RN \\
\midrule\noalign{}
\endfirsthead
\toprule\noalign{}
Region & f\_h & MOS\_h & n\_h & b\_h & k\_h & RN \\
\midrule\noalign{}
\endhead
\bottomrule\noalign{}
\endlastfoot
3 & 0.0321459 & 1552 & 3.049021 & 23.37538 & 2877.0 & 1321 \\
4 & 0.0306724 & 772 & 1.715096 & 19.72323 & 2427.5 & 131 \\
\end{longtable}

\subsection{\texorpdfstring{\emph{Undersized Schools
Linking}}{Undersized Schools Linking}}\label{undersized-schools-linking}

In cases where a selected school had fewer students than the desired
cluster size \(m^*_h\), the within-school sampling rate would exceed
1.0, making the design unfeasible. To address this, and to ensured that
the effective number of completed questionnaires per school meets the
targets \(m^*_h\), we linked undersized schools with nearby schools when
their MOS was less than \(\frac{m^*_h}{r}\), where \(r\) is the expected
student response rate (70\%). This operational rule preserves
feasibility without altering theoretical inclusion probabilities. The
within-school sampling rate remains \(\frac{m^*_h}{MOS_{hi}}\),
maintaining epsem across students. For example, 5 small schools in
Region 4 were linked to form a cluster of 10 meeting the required sample
size of 19.729. Note that no oversize schools were identified that
required splitting, and all selected schools acceptable size or linked
as needed. Appendix 8 presents the within school sampling rate for the
two selected strata in Appendix 7.

\subsection{Student Selection Sample}\label{student-selection-sample}

We implement the same systematic random sampling for the roster example
of schools from Region 7. The randomly sampled middle school was from
Region 7, the MOS for this school was 242, but the actual size is 219.
This is the formula for calculating overall sampling fraction:
\(f_h = \frac{n_h MOS_{hi} }{\sum_{i \in h} MOS{hi} } \frac{m^*_h}{MOS_{hi}}\).
We got the expected sample size of 14.53126, and we rounded it to 15. To
obtain the expected sample size, we first got the second-stage sampling
rate is
\(\text{Sampling Rate} = \frac{m^*_{7}}{MOS_7} = \frac{16.05737}{242} = 0.06635277\),
and then multiplied the rate by the actual sample size.

To get the sampling interval
\(k_{hi} = MOS_{hi} / m^*_{h} = \frac{219}{15}\), we choose a random
starting number between 1 and \(k_{hi}\), which is 14.6, then we use the
k-interval 146 to conduct the systematic sampling. Then we selected the
student at the random start position (14) and every \(k_{hi}\)-th
student thereafter from the ordered roster. The roaster of names is in
Appendix 9.

\section{Estimation Plan}\label{estimation-plan}

For this proposal, we use the paired selection model to help us estimate
variance. However, we cannot form a pseudo-stratum from just one cluster
nor have odd number of clusters, since paired methods require at least
two units within a stratum. A key constraint for this approach is that
each stratum must contain at least two independent variance units. We
collapse strata that have odd number clusters or just one cluster with
the adjacent stratum. In Appendix 7, we show collapsed strata and
corresponding number of sampling error computation units (SECU); we
collapsed regions 1 to 3 into stratum 1, regions 4 to 7 into stratum 2,
and left regions 8 and 9 the same. We take all of the linked clusters
and divide them by two to get the paired SECU count. We now have a total
of 4 strata in this model, 3 SECUs for stratum 1, 103 SECUs for stratum
2, 77 SECUs for stratum 3, and 135 SECUs for stratum 4 resulting in a
total of 318 SECUs across strata. In our estimation plan, each sampled
unit (student) is assigned a stratum code indicating the explicit
sampling stratum from which its PSU (clusters) was drawn.

\subsubsection{\texorpdfstring{\emph{Variance
Estimation}}{Variance Estimation}}\label{variance-estimation}

In this study, we need to estimate the proportion who have ever smoked,
the proportion who have ever used marijuana, and the mean age at first
use of cigarettes or marijuana. We did not consider using weight here
since we maintained epsem design through the whole sampling process,
which means every student has the same probability of being selected.
However, in practice, we need to consider the response rate(the response
rate among schools will be 30 percent, and the response rate among
teenagers within schools will be 70 percent), which we should adjust our
weight in design based on the response rate when conducting variance
estimation. We decided to use Taylor Series Linearization to estimate
the ratio estimator, which is approximated as:
\(\text{For all strata } n_h = 2\).

\[
\text{var}(r) \approx \frac{1}{\hat{t}_x^2} \left[ \sum_h \text{var}(\hat{t}_{h,y}) + r^2 \sum_h \text{var}(\hat{t}_{h,x}) - 2r \sum_h \text{cov}(\hat{t}_{h,y}, \hat{t}_{h,x}) \right]
\]

The general estimator used is the ratio estimator:
\(\hat{r} = \frac{\hat{t}_y}{\hat{t}_x}\)

\begin{itemize}
\item
  \(\hat{t}_y\): estimated total for the numerator variable(e.g., number
  of students who smoked)
\item
  \(\hat{t}_x\): estimated total for the denominator(e.g., total
  eligible students)
\item
  \(\text{var}(\hat{t}_{h,y})\): variance of the numerator total within
  stratum \(h\)
\item
  \(\text{var}(\hat{t}_{h,x})\):variance of the denominator total within
  stratum \(h\)
\item
  \(\text{cov}(\hat{t}{h,y}, \hat{t}{h,x})\): covariance between
  numerator and denominator totals within stratum \(h\)
\end{itemize}

\subsubsection{\texorpdfstring{\emph{Confidence
Interval}}{Confidence Interval}}\label{confidence-interval}

A 95\% confidence interval for the estimated proportion and mean
\(\hat{r}\) is given by:
\(\hat{r} \pm t_{df, 0.975} \cdot \text{SE}(\hat{r})\), where,
\(SE(\hat{r}) = \sqrt{\text{Var}(\hat{r})}\), and the df is equal to the
number of SECUs minus the number of strata, which is 318-4 = 314. We
conducted a simulation of the variance estimation of the proportion of
students who ever smoked by applying the variance estimation formula,
and we got: Ratio Estimator = 0.2505455, SE = 0.00003422286, 95\%
confidence interval {[}0.2504782, 0.2506129{]}.

\subsubsection{\texorpdfstring{\emph{Subclass
Estimation}}{Subclass Estimation}}\label{subclass-estimation}

For the 20\% subgroup, we used the same estimation and variance
formulas, but applied them to a subset of the data that reflects 20\% of
the full population(lower-income households). However, the low-income
students are only 20\% of the population, which means some SECUs may
contain a very small number of low-income students or no low-income
students, thus, may not meet the expected subclass size per cluster (
\(\text{expected }b^* \text{subclass pct} = 76.66519 \times 0.20 = 15.33304\)).
In that case, we might link undersized units to form linked units of the
minimum sufficient size.

In conclusion, this report details a robust and statistically efficient
two-stage stratified cluster sampling design tailored to the MDE's need
for monitoring teenage smoking and drug use. By employing stratification
across educational regions, probability proportional to size selection
for schools, and systematic sampling of students within schools, the
design ensures representative statewide and regional estimates while
adhering to budget constraints. The chosen parameters, including an
anticipated sample of approximately 88 schools and 4,721 students, are
optimized to achieve the required precision for key variables after
accounting for anticipated non-response. The outlined procedures for
selection, including linkage for smaller schools, and the comprehensive
estimation plan provide a clear roadmap for survey implementation and
analysis, ultimately delivering a cost-effective solution capable of
generating the critical data required by the MDE.

\newpage

\section{Appendix 1: Estimating SRS Desired Sample
Size}\label{appendix-1-estimating-srs-desired-sample-size}

For each target variable alongside desired levels of means and sampling
variance we display below the element variance, standard deviation and
standard error, and sampling variance. We use the sampling variance as
the denominator to determine SRS sampling size for each target variable.

\begin{longtable}[]{@{}
  >{\centering\arraybackslash}p{(\columnwidth - 10\tabcolsep) * \real{0.2254}}
  >{\centering\arraybackslash}p{(\columnwidth - 10\tabcolsep) * \real{0.2535}}
  >{\centering\arraybackslash}p{(\columnwidth - 10\tabcolsep) * \real{0.1549}}
  >{\centering\arraybackslash}p{(\columnwidth - 10\tabcolsep) * \real{0.1127}}
  >{\centering\arraybackslash}p{(\columnwidth - 10\tabcolsep) * \real{0.1549}}
  >{\centering\arraybackslash}p{(\columnwidth - 10\tabcolsep) * \real{0.0986}}@{}}
\toprule\noalign{}
\begin{minipage}[b]{\linewidth}\centering
Outcome
\end{minipage} & \begin{minipage}[b]{\linewidth}\centering
Element Variance
\end{minipage} & \begin{minipage}[b]{\linewidth}\centering
SD
\end{minipage} & \begin{minipage}[b]{\linewidth}\centering
SE
\end{minipage} & \begin{minipage}[b]{\linewidth}\centering
Variance
\end{minipage} & \begin{minipage}[b]{\linewidth}\centering
SRS N
\end{minipage} \\
\midrule\noalign{}
\endhead
\bottomrule\noalign{}
\endlastfoot
smoked\_cig & 0.1875 & 0.4330127 & 0.0125 & 0.0001563 & 1200 \\
smoked\_mj & 0.1275 & 0.3570714 & 0.0075 & 0.0000562 & 2267 \\
age\_approached & 1.0000 & 1.0000000 & 0.6000 & 0.3600000 & 3 \\
\end{longtable}

\newpage

\section{Appendix 2: Proportionate Allocation Across
Strata}\label{appendix-2-proportionate-allocation-across-strata}

We compute a proportionate allocation of students across all nine
strata. For instance, 45\% of students across 923 schools in this
population come from stratum 9; therefore, the proportionate allocation
for this stratum is .4514 in the below table.

\begin{longtable}[]{@{}cccc@{}}
\toprule\noalign{}
Region & Total Student & Total Schools & Proportionate Allocation \\
\midrule\noalign{}
\endhead
\bottomrule\noalign{}
\endlastfoot
1 & 3561 & 20 & 0.0042896 \\
2 & 5474 & 30 & 0.0065941 \\
3 & 8631 & 33 & 0.0103971 \\
4 & 4855 & 31 & 0.0058484 \\
5 & 18907 & 80 & 0.0227757 \\
6 & 33133 & 133 & 0.0399126 \\
7 & 191992 & 644 & 0.2312772 \\
8 & 188830 & 549 & 0.2274682 \\
9 & 374755 & 923 & 0.4514370 \\
\end{longtable}

\newpage

\section{Appendix 3: We Use Pilot Study DEFFs to Estimate
roh}\label{appendix-3-we-use-pilot-study-deffs-to-estimate-roh}

We estimate roh form the design effects provided from a similar pilot
study. Below we show roh estimates for each target variable.

\begin{longtable}[]{@{}ccc@{}}
\toprule\noalign{}
Outcome & DEFF & roh \\
\midrule\noalign{}
\endhead
\bottomrule\noalign{}
\endlastfoot
smoked\_cig & 2.5 & 0.0306122 \\
smoked\_mj & 2.0 & 0.0204082 \\
age\_approached & 1.7 & 0.0142857 \\
\end{longtable}

\newpage

\section{Appendix 4: Evaluating Alternative Clustering
Designs}\label{appendix-4-evaluating-alternative-clustering-designs}

The table below presents the three optimum subsample sizes for each
target variable alongside estimates of sampling variance, standard
errors, and a variance check to determine which option is optimal for
this design.

\begin{longtable}[]{@{}
  >{\centering\arraybackslash}p{(\columnwidth - 16\tabcolsep) * \real{0.1702}}
  >{\centering\arraybackslash}p{(\columnwidth - 16\tabcolsep) * \real{0.0851}}
  >{\centering\arraybackslash}p{(\columnwidth - 16\tabcolsep) * \real{0.1064}}
  >{\centering\arraybackslash}p{(\columnwidth - 16\tabcolsep) * \real{0.1064}}
  >{\centering\arraybackslash}p{(\columnwidth - 16\tabcolsep) * \real{0.1064}}
  >{\centering\arraybackslash}p{(\columnwidth - 16\tabcolsep) * \real{0.1064}}
  >{\centering\arraybackslash}p{(\columnwidth - 16\tabcolsep) * \real{0.1170}}
  >{\centering\arraybackslash}p{(\columnwidth - 16\tabcolsep) * \real{0.1170}}
  >{\centering\arraybackslash}p{(\columnwidth - 16\tabcolsep) * \real{0.0851}}@{}}
\toprule\noalign{}
\begin{minipage}[b]{\linewidth}\centering
Outcome
\end{minipage} & \begin{minipage}[b]{\linewidth}\centering
Option
\end{minipage} & \begin{minipage}[b]{\linewidth}\centering
deff\_new
\end{minipage} & \begin{minipage}[b]{\linewidth}\centering
var\_srs
\end{minipage} & \begin{minipage}[b]{\linewidth}\centering
var\_crs
\end{minipage} & \begin{minipage}[b]{\linewidth}\centering
se
\end{minipage} & \begin{minipage}[b]{\linewidth}\centering
lower
\end{minipage} & \begin{minipage}[b]{\linewidth}\centering
upper
\end{minipage} & \begin{minipage}[b]{\linewidth}\centering
var\_ck
\end{minipage} \\
\midrule\noalign{}
\endhead
\bottomrule\noalign{}
\endlastfoot
smoked\_cig & 1 & 2.303745 & 0.000057 & 0.000131 & 0.011464 & 0.227530 &
0.272470 & Y \\
smoked\_mj & 1 & 1.869163 & 0.000030 & 0.000057 & 0.007527 & 0.135248 &
0.164752 & N \\
age\_approached & 1 & 1.608414 & 0.002139 & 0.003441 & 0.058660 &
11.885027 & 12.114973 & Y \\
smoked\_cig & 2 & 2.612213 & 0.000051 & 0.000133 & 0.011525 & 0.227412 &
0.272588 & Y \\
smoked\_mj & 2 & 2.074809 & 0.000027 & 0.000056 & 0.007486 & 0.135327 &
0.164673 & Y \\
age\_approached & 2 & 1.752366 & 0.001907 & 0.003341 & 0.057802 &
11.886707 & 12.113293 & Y \\
smoked\_cig & 3 & 2.939066 & 0.000046 & 0.000136 & 0.011676 & 0.227114 &
0.272886 & Y \\
smoked\_mj & 3 & 2.292711 & 0.000025 & 0.000057 & 0.007517 & 0.135267 &
0.164733 & N \\
age\_approached & 3 & 1.904898 & 0.001740 & 0.003314 & 0.057565 &
11.887172 & 12.112828 & Y \\
\end{longtable}

\newpage

\section{Appendix 5: Strata Cluster Selection, Sampling
Interval}\label{appendix-5-strata-cluster-selection-sampling-interval}

We calculate using the proportionate allocation and n\_opt the number of
clusters to select from each strata. Additionally, we compute sampling
intervals and select random starts from 1 to k.

\begin{longtable}[]{@{}cccc@{}}
\toprule\noalign{}
Region & n\_h & k\_h & RN \\
\midrule\noalign{}
\endhead
\bottomrule\noalign{}
\endlastfoot
1 & 1.257973 & 3561.000 & 3168 \\
2 & 1.933767 & 2737.000 & 2310 \\
3 & 3.049021 & 2877.000 & 1321 \\
4 & 1.715096 & 2427.500 & 131 \\
5 & 6.679161 & 2701.000 & 2122 \\
6 & 11.704693 & 2761.083 & 2114 \\
7 & 67.823846 & 2823.412 & 374 \\
8 & 66.706826 & 2818.358 & 380 \\
9 & 132.387420 & 2839.053 & 1673 \\
\end{longtable}

\newpage

\section{Appendix 6: Linkage of Schools for Stage 1
Selection}\label{appendix-6-linkage-of-schools-for-stage-1-selection}

We show the number of clusters to be sampled for each strata along with
the number of total linked schools and total number of students in each
strata to use for random selection in the second stage.

\begin{longtable}[]{@{}cccc@{}}
\toprule\noalign{}
Region & Clusters & Schools & MOS \\
\midrule\noalign{}
\endhead
\bottomrule\noalign{}
\endlastfoot
1 & 1 & 1 & 430 \\
2 & 2 & 2 & 933 \\
3 & 3 & 3 & 1552 \\
4 & 10 & 14 & 772 \\
5 & 7 & 7 & 4400 \\
6 & 12 & 12 & 8202 \\
7 & 177 & 281 & 52360 \\
8 & 154 & 224 & 59827 \\
9 & 270 & 379 & 141443 \\
\end{longtable}

\newpage

\section{Appendix 7: Evaluating Alternative Clustering
Designs}\label{appendix-7-evaluating-alternative-clustering-designs}

We present a table with the following pseudo strata that were combined
to used paired sampling, as well as the number SECUs associated.

\begin{longtable}[]{@{}cc@{}}
\toprule\noalign{}
Psuedo & SECU \\
\midrule\noalign{}
\endhead
\bottomrule\noalign{}
\endlastfoot
1 & 3 \\
2 & 103 \\
3 & 77 \\
4 & 135 \\
\end{longtable}

\newpage

\section{Appendix 8: Within-School Sampling Interval for 2
Strata}\label{appendix-8-within-school-sampling-interval-for-2-strata}

We show the selection of schools below after selection for two strata
and within-school sampling rate. The within-school sampling interval was
calculated as the total student count divided by the target sample size
(m\_h\_star). Students were then selected systematically using a random
start between 1 and the interval.

\begin{longtable}[]{@{}
  >{\centering\arraybackslash}p{(\columnwidth - 8\tabcolsep) * \real{0.2353}}
  >{\centering\arraybackslash}p{(\columnwidth - 8\tabcolsep) * \real{0.1324}}
  >{\centering\arraybackslash}p{(\columnwidth - 8\tabcolsep) * \real{0.1176}}
  >{\centering\arraybackslash}p{(\columnwidth - 8\tabcolsep) * \real{0.1618}}
  >{\centering\arraybackslash}p{(\columnwidth - 8\tabcolsep) * \real{0.3529}}@{}}
\toprule\noalign{}
\begin{minipage}[b]{\linewidth}\centering
Selection Num.
\end{minipage} & \begin{minipage}[b]{\linewidth}\centering
Stratum
\end{minipage} & \begin{minipage}[b]{\linewidth}\centering
School
\end{minipage} & \begin{minipage}[b]{\linewidth}\centering
Cumul MOS
\end{minipage} & \begin{minipage}[b]{\linewidth}\centering
Within School Interval
\end{minipage} \\
\midrule\noalign{}
\endhead
\bottomrule\noalign{}
\endlastfoot
4 & 3 & 01155 & 928 & 56.714138 \\
5 & 3 & 01527 & 1355 & 26.095837 \\
6 & 3 & 04860 & 1552 & 12.039531 \\
7 & 4 & 02692 & 323 & 23.395186 \\
8 & 4 & 06812 & 387 & 4.635579 \\
9 & 4 & 08446 & 444 & 4.128562 \\
10 & 4 & 08063 & 493 & 3.549115 \\
11 & 4 & 03998 & 540 & 3.404253 \\
12 & 4 & 04034 & 586 & 3.331822 \\
13 & 4 & 09308 & 629 & 3.114529 \\
14 & 4 & 02305 & 670 & 2.969668 \\
15 & 4 & 08521 & 703 & 2.390220 \\
16 & 4 & 07124 & 723 & 4.997733 \\
\end{longtable}

\newpage

\section{Appendix 9: Students Selected From One
School.}\label{appendix-9-students-selected-from-one-school.}

\begin{longtable}[]{@{}ccccc@{}}
\caption{Evaluating Alternative Clustering Designs}\tabularnewline
\toprule\noalign{}
ID & Grade & class & Firstname & Lastname \\
\midrule\noalign{}
\endfirsthead
\toprule\noalign{}
ID & Grade & class & Firstname & Lastname \\
\midrule\noalign{}
\endhead
\bottomrule\noalign{}
\endlastfoot
1 & 7 & Grady Vest & Magee & Monica L \\
16 & 7 & Grady Vest & Schwartz & David Scott \\
30 & 7 & Qixuan Li & Raynor & Gregory K \\
45 & 7 & Qixuan Li & Franke & Mira A \\
59 & 7 & Bill Pesau & Black & Stephen P \\
74 & 7 & Andrew Bellman & Marbut & Joanne Renee \\
89 & 7 & Andrew Bellman & Trecker & Molly A \\
103 & 8 & Joe Williams & Dawson & Rebecca S \\
118 & 8 & Joe Williams & Krosky & Paula Michele \\
132 & 8 & Robert Mcfay & O Brien & Erin Terese \\
147 & 8 & Joseph Miden & Kosove & Daniel Brian \\
162 & 8 & Joseph Miden & Loevy & Debra L \\
176 & 8 & James Vogner & Dworzanowski & Gregory William \\
191 & 8 & James Vogner & Gatto & Julia Lynn \\
205 & 8 & Jane Doe & Kaye & Peter Mitchell \\
\end{longtable}




\end{document}
