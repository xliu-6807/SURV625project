% Options for packages loaded elsewhere
\PassOptionsToPackage{unicode}{hyperref}
\PassOptionsToPackage{hyphens}{url}
\PassOptionsToPackage{dvipsnames,svgnames,x11names}{xcolor}
%
\documentclass[
  12pt]{article}

\usepackage{amsmath,amssymb}
\usepackage{iftex}
\ifPDFTeX
  \usepackage[T1]{fontenc}
  \usepackage[utf8]{inputenc}
  \usepackage{textcomp} % provide euro and other symbols
\else % if luatex or xetex
  \usepackage{unicode-math}
  \defaultfontfeatures{Scale=MatchLowercase}
  \defaultfontfeatures[\rmfamily]{Ligatures=TeX,Scale=1}
\fi
\usepackage{lmodern}
\ifPDFTeX\else  
    % xetex/luatex font selection
\fi
% Use upquote if available, for straight quotes in verbatim environments
\IfFileExists{upquote.sty}{\usepackage{upquote}}{}
\IfFileExists{microtype.sty}{% use microtype if available
  \usepackage[]{microtype}
  \UseMicrotypeSet[protrusion]{basicmath} % disable protrusion for tt fonts
}{}
\makeatletter
\@ifundefined{KOMAClassName}{% if non-KOMA class
  \IfFileExists{parskip.sty}{%
    \usepackage{parskip}
  }{% else
    \setlength{\parindent}{0pt}
    \setlength{\parskip}{6pt plus 2pt minus 1pt}}
}{% if KOMA class
  \KOMAoptions{parskip=half}}
\makeatother
\usepackage{xcolor}
\setlength{\emergencystretch}{3em} % prevent overfull lines
\setcounter{secnumdepth}{5}
% Make \paragraph and \subparagraph free-standing
\makeatletter
\ifx\paragraph\undefined\else
  \let\oldparagraph\paragraph
  \renewcommand{\paragraph}{
    \@ifstar
      \xxxParagraphStar
      \xxxParagraphNoStar
  }
  \newcommand{\xxxParagraphStar}[1]{\oldparagraph*{#1}\mbox{}}
  \newcommand{\xxxParagraphNoStar}[1]{\oldparagraph{#1}\mbox{}}
\fi
\ifx\subparagraph\undefined\else
  \let\oldsubparagraph\subparagraph
  \renewcommand{\subparagraph}{
    \@ifstar
      \xxxSubParagraphStar
      \xxxSubParagraphNoStar
  }
  \newcommand{\xxxSubParagraphStar}[1]{\oldsubparagraph*{#1}\mbox{}}
  \newcommand{\xxxSubParagraphNoStar}[1]{\oldsubparagraph{#1}\mbox{}}
\fi
\makeatother


\providecommand{\tightlist}{%
  \setlength{\itemsep}{0pt}\setlength{\parskip}{0pt}}\usepackage{longtable,booktabs,array}
\usepackage{calc} % for calculating minipage widths
% Correct order of tables after \paragraph or \subparagraph
\usepackage{etoolbox}
\makeatletter
\patchcmd\longtable{\par}{\if@noskipsec\mbox{}\fi\par}{}{}
\makeatother
% Allow footnotes in longtable head/foot
\IfFileExists{footnotehyper.sty}{\usepackage{footnotehyper}}{\usepackage{footnote}}
\makesavenoteenv{longtable}
\usepackage{graphicx}
\makeatletter
\def\maxwidth{\ifdim\Gin@nat@width>\linewidth\linewidth\else\Gin@nat@width\fi}
\def\maxheight{\ifdim\Gin@nat@height>\textheight\textheight\else\Gin@nat@height\fi}
\makeatother
% Scale images if necessary, so that they will not overflow the page
% margins by default, and it is still possible to overwrite the defaults
% using explicit options in \includegraphics[width, height, ...]{}
\setkeys{Gin}{width=\maxwidth,height=\maxheight,keepaspectratio}
% Set default figure placement to htbp
\makeatletter
\def\fps@figure{htbp}
\makeatother

\addtolength{\oddsidemargin}{-.5in}%
\addtolength{\evensidemargin}{-1in}%
\addtolength{\textwidth}{1in}%
\addtolength{\textheight}{1.7in}%
\addtolength{\topmargin}{-1in}%
\makeatletter
\@ifpackageloaded{caption}{}{\usepackage{caption}}
\AtBeginDocument{%
\ifdefined\contentsname
  \renewcommand*\contentsname{Table of contents}
\else
  \newcommand\contentsname{Table of contents}
\fi
\ifdefined\listfigurename
  \renewcommand*\listfigurename{List of Figures}
\else
  \newcommand\listfigurename{List of Figures}
\fi
\ifdefined\listtablename
  \renewcommand*\listtablename{List of Tables}
\else
  \newcommand\listtablename{List of Tables}
\fi
\ifdefined\figurename
  \renewcommand*\figurename{Figure}
\else
  \newcommand\figurename{Figure}
\fi
\ifdefined\tablename
  \renewcommand*\tablename{Table}
\else
  \newcommand\tablename{Table}
\fi
}
\@ifpackageloaded{float}{}{\usepackage{float}}
\floatstyle{ruled}
\@ifundefined{c@chapter}{\newfloat{codelisting}{h}{lop}}{\newfloat{codelisting}{h}{lop}[chapter]}
\floatname{codelisting}{Listing}
\newcommand*\listoflistings{\listof{codelisting}{List of Listings}}
\makeatother
\makeatletter
\makeatother
\makeatletter
\@ifpackageloaded{caption}{}{\usepackage{caption}}
\@ifpackageloaded{subcaption}{}{\usepackage{subcaption}}
\makeatother

\ifLuaTeX
  \usepackage{selnolig}  % disable illegal ligatures
\fi
\usepackage[]{natbib}
\bibliographystyle{agsm}
\usepackage{bookmark}

\IfFileExists{xurl.sty}{\usepackage{xurl}}{} % add URL line breaks if available
\urlstyle{same} % disable monospaced font for URLs
\hypersetup{
  pdftitle={Michigan Teen Smoking and Drug Use Survey Sample Design},
  pdfauthor={Kevin Linares; Jianing Zou; Weishan Jiang; Xiaoqing Liu},
  colorlinks=true,
  linkcolor={blue},
  filecolor={Maroon},
  citecolor={Blue},
  urlcolor={Blue},
  pdfcreator={LaTeX via pandoc}}



\begin{document}


\def\spacingset#1{\renewcommand{\baselinestretch}%
{#1}\small\normalsize} \spacingset{1}


%%%%%%%%%%%%%%%%%%%%%%%%%%%%%%%%%%%%%%%%%%%%%%%%%%%%%%%%%%%%%%%%%%%%%%%%%%%%%%

\date{April 17, 2025}
\title{\bf Michigan Teen Smoking and Drug Use Survey Sample Design}
\author{
Kevin Linares\\
and\\Jianing Zou\\
and\\Weishan Jiang\\
and\\Xiaoqing Liu\\
University of Maryland\\
}
\maketitle

\bigskip
\bigskip
\begin{abstract}

\end{abstract}


\newpage
\spacingset{1.9} % DON'T change the spacing!


\section{Introduction}\label{sec-intro}

The State of Michigan Department of Education requires data to monitor
teenage smoking and drug use, partly to assess compliance with tobacco
industry settlements. This report details the design of a statewide
probability sample of Michigan teenagers (enrolled students in grades
7-12) developed to meet the Department's needs. The objective was to
create a cost-effective sample design capable of producing estimates for
key variables with a coefficient of variation (CV) no greater than 0.05,
for both the state overall and for specific regions. Based on cost
considerations and a review of alternatives, a two-stage school-based
sample design was chosen by the client. This report outlines the overall
design, stratification and allocation plan, selection procedures, and
estimation methods.

\section{Overall Design}\label{sec-meth}

The sample design is a two-stage stratified cluster sample.

\begin{itemize}
\item
  \textbf{Target Population:} Students enrolled in grades 7 through 12
  in public and non-public schools in Michigan during the fall 2025
  survey period. The estimated total population size is N = 830,138
  students.
\item
  \textbf{Sampling Frame:} The primary frame is the Michigan Department
  of Education's 2024 list of public and non-public schools with student
  headcounts. This frame is assumed to be complete for the target
  population, although it excludes homeschooled students and those who
  have dropped out. At the second stage, student rosters will be
  obtained from selected schools.
\item
  \textbf{Stages of Selection:}
\end{itemize}

\begin{verbatim}
-   Stage 1 (Primary Sampling Units, PSUs): Schools selected with probability proportional to size (PPeS), where size is the school's total student headcount (`tot_all`).

-   Stage 2 (Secondary Sampling Units, SSUs): Students selected systematically from rosters within selected schools.
     
\end{verbatim}

\begin{itemize}
\item
  \textbf{Stratification:} The frame of schools is stratified explicitly
  by the nine education regions defined by the client to improve
  precision and allow for regional estimates. Implicit stratification is
  achieved by sorting schools within each region prior to systematic
  selection (e.g., by size).
\item
  \textbf{Key Variables \& Precision Targets:} The design focuses on
  achieving a coefficient of variation (CV) of 0.05 or less for
  state-level estimates of:
\end{itemize}

\begin{verbatim}
-    Proportion who ever smoked one cigarette (expected p = 0.25).

-   Proportion who ever smoked marijuana (expected p = 0.15).

-   Mean age when first approached to smoke cigarettes or marijuana (expected mean = 12, expected SD = 1).
\end{verbatim}

\begin{itemize}
\item
  \textbf{Initial SRS Sample Sizes:} Ignoring clustering and FPC, the
  required SRS sample sizes to meet the CV=0.05 target were calculated
  as: n\_srs(cig)=1200, n\_srs(mj)=2267, and n\_srs(age)=3.
\item
  \textbf{Design Parameters (roh, deff):} Using design effects from a
  similar prior study (deff\_cig=2.5, deff\_mj=2.0, deff\_age=1.7, based
  on a=150 schools, m=50 students/school), the estimated intraclass
  correlations (roh) were calculated as: roh\_cig=0.0306,
  roh\_mj=0.0204, roh\_age=0.0143.
\item
  \textbf{Optimal Cluster Size (b*) and Number of Clusters (a):}
  Considering a budget of \$500,000 and estimated costs of C\_a=\$3,000
  per school and C\_b=\$50 per student, the optimal subsample size (b*,
  denoted m\_opt in code) and number of schools (a, denoted n\_opt in
  code) were calculated for each variable. To meet the precision target
  for all variables within budget, the design based on optimizing for
  marijuana smoking (which required the largest sample size) was chosen.
  This yields:

  \begin{itemize}
  \item
    Optimal average cluster size: \textbf{b* = 53 students}
  \item
    Optimal number of clusters (schools): \textbf{a = 87 schools} This
    design has an estimated total cost of \$491,550.
  \end{itemize}
\item
  \textbf{Non-Response Adjustments:} To account for anticipated
  non-response rates of 70\% among schools (30\% RR) and 30\% among
  students within cooperating schools (70\% RR), the number of schools
  and the within-school target take were adjusted:

  \begin{itemize}
  \item
    Number of schools to select: a\_select = 87 / 0.30 ≈ \textbf{290
    schools}
  \item
    Target students per school (for fieldwork): b\_field = 53 / 0.70 ≈
    \textbf{76 students}
  \end{itemize}
\end{itemize}

Stratification and Allocation

The sample of schools is explicitly stratified by the nine education
regions provided by the client. The allocation of the sample across
these strata is designed to be proportionate to the total number of
students (Measure of Size, MOS) in each stratum, aiming for an equal
probability of selection method (epsem) overall.




\end{document}
